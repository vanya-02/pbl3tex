\section{Domain Analysis}

\subsection{Impact of the problem over the domain of study}
\par lorem

\subsection{Target group}
\par The leading audience of the app will be English speaking active study-working people which have a lot of tasks events which are related to teams or companies that work on different platforms, or for startuppers who have an extremely flexible schedule and need a platform which will help to organize all particular tasks and events on one board. The audience target age is 16-35 

\subsection{Customer validation}
\par lorem

\subsection{Competition}
\par 
\begin{enumerate}
	\item 	Google calendar is not the most intuitive app of using tasks and schedules it has a lot of special conditions to get the desired result 
	
	\item SuperSaaS is good at creating group schedule, but because of the design of the site is hard to understand what happens, saves the situation the support and tutorials page, it supports the work with forms directly when accessing the schedule which is very efficient in scope to get additional information  

	\item Outlook to get all the possibilities of the calendar you need to pay, is free for students and have many integrated applications which can communicate direct whit each other that increase productivity it is a combination of calendar and email it is very easy to share about events and set tasks and reminders for groups or selected people 

	\item Jorte Calendar theme changing, can use free but to get the chance to change the appearance of the calendar you need to pay subscription  

	\item Calendly it is app which works very good with establishing new meetings and events because it gave you to choose the rules that all participants can choose between the permission you establish it is time zone but the free users are limited to 1 calendar and basic functions which do not let user to feel the power of this app  
\end{enumerate}				

\clearpage