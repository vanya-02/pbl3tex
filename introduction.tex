	\section{Introduction}

\subsection{Domain analysis}
\subsubsection{Abstract}
\par Domain analysis is the process by which a software engineer learns background information. Our team had to learn sufficient information so as to be able to understand the problem and make good decisions during requirements analysis and other stages of the software engineering process. The word ‘domain’ in our case means the general field of business or technology in which the customers expect to be using the software. 

Some domains might be very broad, such as ‘airline reservations’, ‘medical diagnosis’, and ‘financial analysis’. Others are narrower, such as ‘the manufacturing of paint’ or ‘scheduling meetings’. People who work in a domain and who have a deep knowledge of it (or part of it), are called domain experts. Many of these people may become customers or users. 

In order to best understand our domain and develop a wider understanding of the task overall our team has done an extended research regarding the topic at hand. 

In order to perform a good analysis as already mentioned, we gathered information from whatever sources of information were available: including domain experts; any books about the domain; any existing software and its documentation, and any other documents we could find. The interviewing, brainstorming and use case analysis techniques discussed later in this chapter can be of assistance with the domain analysis.  

As a software engineer, you are not expected to become an expert in the domain; nevertheless, domain analysis can involve considerable work. Benefits such as Faster development, Better system understanding and an Anticipation of extension will be of great help for future development and make the work worthwhile. 
\subsection{Problem formulation}
\par Digital calendars are a mess. They're a crucial part of modern life, especially as remote work becomes more prominent. They help employees/students to organize their day and bosses/managers can actually make sense of the work that's getting done. However too often, most of this work is just about … dealing with calendars. Employees/students fill all their half-hour boxes with tasks, meetings and personal commitments, only to have bosses, clients, studies and co-workers/colleagues steal that time, one unexpected invite at a time. Once-focused days turn into a haphazard series of too-long meetings and too-short breaks, with little time to get actual work done. 

Therefore, for our project we decided to tackle the problem of the big quantity of the calendars, reminders, task apps, in day-by-day life. This is a big problem that we encounter and this is what we base on. 

From our experience, we regularly get tasks which contains deadlines. The problem is that we get them on different platforms from several organizations like university, work, study centers or other places. We get many daily events which take place on separate platforms. We can take and rewrite and introduce the same information multiple times on different calendars and this will take a lot of time and efforts. And all this can increase the chance of error occurrence, much less people that are working with many clients and customers which can daily organize different events on different platforms. They deal with focus loss and this can influence their and other lives like a forgotten meeting, undone work or missed deadline.  

\subsection{Solution concept}
\par An online platform designed for vast groups of users for day-by-day life activities to more specific workflows and study schedules. The platform aims to have a calendar with a list of tasks and a lot of modules that can be chosen and integrated by choice as desired by anyone. Users will have the option to connect from a catalog of helpful modules which will then be integrated into the main calendar app for a better and more personalized user experience. Like a FAQ tab or a ‘get-started’ option, we will plan to introduce a platform-wide smart-assistant that will help the user step-by-step how to utilize the platform. The end goal is a platform that will help people to organize time, as combining many platforms in one will improve work speed and will help with reminders which will make planning effortless and a lot more enjoyable than using the traditional organizational application stacks.   

Also, and also our second aim of this platform is to have all schedules arranged on one screen where users can easily interact with them as well due to the introduced assistant, interaction with the platform will be more interactive and intuitive. We want users to be able to manage their time-efficient using our platform, due to installed custom modules there are no limits on the site's abilities in terms of interaction with users, it can show incoming tasks, show the traffic situation, shows the bus in the traffic, talk about weather conditions and recommendation about the weather. Also, it can communicate with other sites via API (Application Programming Interface) (for syncing purposes) to increase even more the possibilities of user managing time using our platform like booking a place to a café, make invitations for events, booking seats to the movie theatre.  

Furthermore, to motivate users to use all possibilities of the platform as well as experiment with new features we will introduce a challenges/ranking system based on how much they use all features of the web app.  

Not to get too messy at the start assistant will companion users through all necessary steps to understand how to integrate and use modules as well as how to connect and manage new calendars and how to use them all the potential of the platform.  

\subsection{Motivation}
\par 
General Motivation 

In general, this calendar will help to arrange and scheduling the tasks meeting and all daily routine so users can see how they spent time and how much time they have to do a type of work, included assistant will help to choose what tasks to do next, just introducing all in form of a list of tasks, daily routines and meetings all with time intervals, and the calendar will fill the gaps in calendars confirm the introduced list, in time of creating the list the can be determined a bunch of characteristics of this task its duration, time buffer for getting to a place and so on. Besides, if two events are colliding with each other the calendars will ask these two events to move to another time or simply delete it. 

Personal Motivation 

As a student, we have to plan study schedules where we can concentrate on studying so because we get enormous amounts of homework to do, we have to plan and schedule all work hours so we can make all the work done but because we do not get all the task in one time we have to plan and be attentive with time scheduling on all sorts of things. For instance, in one day, we can get a bunch of tasks from university or from work we can get nothing this irregularity of getting tasks can get complicated to schedule moreover if it is something new, we can't exactly say how much time this will cost to do so, we have to be very flexible with switching tasks and to be present at all seminars and laboratories. As students, we want to make a platform that will make a flexible list where we can fast and easily schedule our next day-week of work. 

\clearpage